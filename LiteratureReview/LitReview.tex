\documentclass [12pt]{article}
\usepackage{graphicx} % Required for inserting images
\usepackage{geometry}
\usepackage{pdfpages}
\usepackage{url}
\usepackage{parskip}
\usepackage{listings}
\geometry{
	tmargin=15mm,
	lmargin=15mm,
	rmargin=15mm,
	bmargin=15mm
}
\title{3YP:}
\author{
	Azhagesh, Ashwinkrishna\\
	\texttt{aa9g22@soton.ac.uk}
}
\begin{document}
\maketitle 

\section{Literature Review: }

\subsection{Deriving Physical Laws with Neural Networks: }

All theories are discovered on the shoulders on other physics theories, this naturally introduces bias in the data, and therefore 
this might not be the most natural and simple way to describe experimental data.\\ 

Can we find natural explanations, using AI and humans are biased.\\ 

"This project attempts to take the next step in this field by tackling the problem of the double pendulum, a noto-
riously complex system for its chaotic behaviour, while minimizing the amount of assumptions of the physical
system imposed on the NN. "\\

They use Python3 and neural networks.\\ 

SciNet is a neural network designed to solve physics problems, maybe look into this, and see if it useful in helping with a quick setup.\\ 

Small angles are less than 15 degrees.\\ 

Seperate small angles from larger ones which tend to be more chaotic.\\ 

http://uu.diva-portal.org/smash/get/diva2:1756655/FULLTEXT01.pdf\\ 

Ending was they tried to make this as geeral as possible which led to high order polynomials, which in turn slowed down the derivation, and approximations, hence required more computational power.\\ 

  
\subsection{Distilling Free-Form Natural Laws from Experimental Data: }

Will print it out and read it, so it's easier on the eyes.\\ 

Ah well can't find the printer :(((\\

This is actually incredible, without prior knowledge, complex systems and laws were derivied.\\ 

It became faster at deriving laws as it understood the "alphabet," used to describe those systems.\\ 


They use symbolic regression for this, which I suppose makes sense. \\ 

I can probably use these results and experimental data, and simulate this again. \\

"By combining the chaotic data with
low-velocity in-phase oscillation data, the algo-
rithm converged onto the precise energy laws
after several hours of computation." - This is indeed very interesting.\\ 

"The algorithm’s search is readily parallelizable,
as many candidate functions need to be evaluated
simultaneously. In a 32-core implementation, the
time required ranged from a few minutes for the
harmonic oscillator to 30 hours for the double
pendulum." -- I'm starting to like this, can easily do this quicker on my server.\\ 

"Bootstrapping the double-pendulum search in
this fashion reduced the search time by nearly an
order of magnitude, from 30 to 40 hours of com-
putation to 7 to 8 hours (Fig. 4B)." - Can definitely talk a bit about this, and explore methods to optimise and make this process more efficient.\\ 

"Might this process diminish the role of future
scientists? Quite the contrary: Scientists may use
processes such as this to help focus on interesting
phenomena more rapidly and to interpret their
meaning." - This kills the art though, and art is important.\\ 

https://cdanfort.w3.uvm.edu/courses/237/schmidt-lipson-2009.pdf\\

\subsection{Discovering State Variables Hidden in High Dimensional Data," Brunton, Proctor, Kutz, Proceedings of the National Academy of Sciences, 2016
* Influence: }

Identifying the variables involved in higher dimensional data through video to avoid human bias.\\ 

All physical laws are based on maths, and therefore to dervive new physical laws, the variables must be known.\\ 

"
For example, it took civilizations millennia to formalize basic mechanical variables such
as mass, momentum and acceleration. Only once these notions were formalized, could laws
of mechanical motion be discovered. " - yet A.I. can do what took humans millenia in a matter of minutes, and this is with fake intelligence.\\ 


Is the very way symbolic regression designed, allow it to dervie physical laws with ease? \\ 


"Without the proper state variables, even a simple system may appear enigmatically complex"  - completely agree, without mass and momentum, how can 

For example, a camera observing a swinging pendulum with an imaging resolution of 128 ×
128 pixels in three color channels, will measure 49,152 variables per frame. Yet this enormous
set of measurement, while intuitively descriptive, is neither compact nor complete: In fact, we
know that the state of a swinging pendulum can be described fully by only two variables: its
angle and angular velocity. Moreover these two state variables cannot be measured from a single
video frame alone. In other words, a single frame, despite the large number of measurements,
is insufficient to describe the full state of a pendulum. - I agree, this makes sense I suppose.\\ 


https://arxiv.org/pdf/2112.10755\\ 

This is a good paper to go through slowly and throughly.\\ 











\section{Methods and tools: }
Dynamic
Mode Decomposition (DMD) and Singular Value Decomposition (SVD) (4)\\
SciNet\\
PySR\\
Python3\\
 Feynman Lecture on Physics\\
Geometric manifold learning algorithms\\
 Levina-Bickel’s algorithm \\























\end{document}

