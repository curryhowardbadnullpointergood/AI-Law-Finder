\documentclass [12pt]{article}
\usepackage{graphicx} % Required for inserting images
\usepackage{geometry}
\usepackage{pdfpages}
\usepackage{url}
\usepackage{parskip}
\usepackage{listings}
\geometry{
	tmargin=15mm,
	lmargin=15mm,
	rmargin=15mm,
	bmargin=15mm
}
\title{3YP:}
\author{
	Azhagesh, Ashwinkrishna\\
	\texttt{aa9g22@soton.ac.uk}
}
\begin{document}
\maketitle 

\section{Literature Review: }

\subsection{Deriving Physical Laws with Neural Networks: }

All theories are discovered on the shoulders on other physics theories, this naturally introduces bias in the data, and therefore 
this might not be the most natural and simple way to describe experimental data.\\ 

Can we find natural explanations, using AI and humans are biased.\\ 

"This project attempts to take the next step in this field by tackling the problem of the double pendulum, a noto-
riously complex system for its chaotic behaviour, while minimizing the amount of assumptions of the physical
system imposed on the NN. "\\

They use Python3 and neural networks.\\ 

SciNet is a neural network designed to solve physics problems, maybe look into this, and see if it useful in helping with a quick setup.\\ 

Small angles are less than 15 degrees.\\ 

Seperate small angles from larger ones which tend to be more chaotic.\\ 

http://uu.diva-portal.org/smash/get/diva2:1756655/FULLTEXT01.pdf\\ 

Ending was they tried to make this as geeral as possible which led to high order polynomials, which in turn slowed down the derivation, and approximations, hence required more computational power.\\ 




\end{document}

