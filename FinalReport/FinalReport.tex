\documentclass{article}
\usepackage{graphicx}
\usepackage{xcolor}
\usepackage[margin=3cm]{geometry} % margins might need to change in the future check with professor adam 
\setlength{\parindent}{0pt}
\usepackage{pgfgantt}

\usepackage{times}
\usepackage{fancyhdr,graphicx,amsmath,amssymb}
\usepackage[ruled,vlined]{algorithm2e}
\include{pythonlisting}


\begin{document}

\begin{center}
    \Large \textcolor{red}{\textbf{Electronics and Computer Science} \\[0.1cm]} 

    \large \textcolor{red}{Faculty of Engineering and Physical Sciences \\[0.1cm]} 

    \textcolor{red}{University of Southampton \\[1cm]} 

    \vspace{1cm} 

    \textcolor{red}{\textbf{\large Ashwinkrishna Azhagesh} \\[0.5cm]}

    \textbf{25/03/2025} \\[1cm] 

    \textbf{\large An AI Approach to Chaotic Physical Systems: } \\[1cm]

    \vspace{0.5cm}

    Project supervisor: \textbf{Adam Peugeot} \\[0.3cm] 
    Second examiner: \textbf{David Millard} \\[1cm]

    Progress report submitted for the award of \\[0.1cm]

    \textbf{\large Bachelors of Science} 
\end{center}

\newpage

{\Huge \textbf{Abstract}}\\[1cm]

Empirical laws are mathematical generalisations found through observing the physical world. It has taken us centuries of gathering data, keen research along with repeated experiments, and no doubt plenty of talented scientists to discover these laws. Leading us to understand everything from the mysteries that govern the collision of two objects to the shape of the path planets thread upon.\\

Recent advances in neural networks including increases in computational power permit us to train models, that replicate, fasten and automate our discovery of empirical laws. This extends to even noisy chaotic systems such as the double pendulum. Combined with white box models, symbolic regression and explanable A.I., we can peer into the "mind," of how such models, process data and conclude their observations. Human congition is inherently finite in its capacity for thought and observational ability, has been historically overcome through the development of new tools such as the microscope. Similarly, congitive biases can be mitigated, by utillising artificial intelligence, which is a rapidly emerging technology capable of expanding our perception and analysis.\\       


\newpage

\fbox{\underline{\textbf{Statement of Originality}}}
\\[0.5cm]

- I have read and understood the ECS Academic Integrity information and the University’s
Academic Integrity Guidance for Students.\\

- I am aware that failure to act in accordance with the Regulations Governing Academic Integrity
may lead to the imposition of penalties which, for the most serious cases, may include
termination of programme.\\

- I consent to the University copying and distributing any or all of my work in any form and
using third parties (who may be based outside the EU/EEA) to verify whether my work
contains plagiarised material, and for quality assurance purposes.\\

\fbox{
\underline{\textbf{You must change the statements in the boxes if you do not agree with them.}\\
}}
\\[0.5cm]

We expect you to acknowledge all sources of information (e.g. ideas, algorithms, data) using
citations. You must also put quotation marks around any sections of text that you have copied
without paraphrasing. If any figures or tables have been taken or modified from another source,
you must explain this in the caption and cite the original source.\\

\fbox{
\underline{\textbf{I have acknowledged all sources, and identified any content taken from elsewhere.}\\
}}
\\[0.5cm]

If you have used any code (e.g. open-source code), reference designs, or similar resources that
have been produced by anyone else, you must list them in the box below. In the report, you must
explain what was used and how it relates to the work you have done.\\


\fbox{
\underline{\textbf{I have not used any resources produced by anyone else.}\\
}}
\\[0.5cm]

You can consult with module teaching staff/demonstrators, but you should not show anyone else
your work (this includes uploading your work to publicly-accessible repositories e.g. Github, unless
expressly permitted by the module leader), or help them to do theirs. For individual assignments,
we expect you to work on your own. For group assignments, we expect that you work only with
your allocated group. You must get permission in writing from the module teaching staff before
you seek outside assistance, e.g. a proofreading service, and declare it here.\\

\fbox{
\underline{\textbf{I did all the work myself, or with my allocated group, and have not helped anyone else.}\\
} }
\\[0.5cm]

We expect that you have not fabricated, modified or distorted any data, evidence, references,
experimental results, or other material used or presented in the report. You must clearly describe
your experiments and how the results were obtained, and include all data, source code and/or
designs (either in the report, or submitted as a separate file) so that your results could be
reproduced.\\

\fbox{
\underline{\textbf{The material in the report is genuine, and I have included all my data/code/designs.}\\
}}
\\[0.5cm]

We expect that you have not previously submitted any part of this work for another assessment.
You must get permission in writing from the module teaching staff before re-using any of your
previously submitted work for this assessment.\\

\fbox{
\underline{\textbf{I have not submitted any part of this work for another assessment.}\\
}}
\\[0.5cm]

If your work involved research/studies (including surveys) on human participants, their cells or
data, or on animals, you must have been granted ethical approval before the work was carried
out, and any experiments must have followed these requirements. You must give details of this in
the report, and list the ethical approval reference number(s) in the box below.\\

\fbox{
\underline{\textbf{My work did not involve human participants, their cells or data, or animals.}\\
}}
\\[0.5cm]


ECS Statement of Originality Template, updated August 2018, Alex Weddell aiofficer@ecs.soton.ac.uk\\


\newpage 

{\Huge \textbf{Abstract}}\\[1cm]


I would like to thank my supervisors, Professor Adam Peugeot and Professor David Millard, for all the help and advise I received throughout this project. \\


\newpage

\tableofcontents 

\newpage
\addcontentsline{toc}{section}{Abstract}
\addcontentsline{toc}{section}{Statement of Originality}



\section{Introduction: }

\subsection{Motivation: }

As there is more data being generated than ever before and new experiments, we need a systematic and au-
tomatic way to deduce various mathematical patterns and laws in these data. Through the use of symbolic
regression we can utilise these data, and in an explainable manner deduce various new physical laws. In this
research I have also extended this beyond physics and have applied this to biological data sets which is a novel
application of this method. Perhaps extend this beyond or add a sectionsaying this can also be applied to nlp
and that it can learn the rules in language and writing etc.
Talk a little about the way this is used outside of this niche use case, and in research, so of course I need to look
and research into this.\\

\section{Previos Work: }

\subsection{Literature Review: }

\section{Noise: }

In this section, I aimed to explore how noise affects the model, and potential ways to mitigate it. Continuing
onwards from the previous model, in the data generation step, noise was artificially added, and the results were
observed.\\

So in order to model the noise, I used the python random library, and generated random numbers between 0 and
an ever increasing amount of randomness, in oder to guage the accuracy as noise increased for the model. I was
also part\\

\subsubsection{How noise affects the model: }
So in order to add noise to the generated data set, I imported in random, and used the randn.int function. In order
to vary the inputs, another function was created that incrememntally passes in higher numbers as parameters
to the random function, allowing each set of generated data to incrememntally become more and more noisy.
Then the symbolic regression model is run on these new data sets, and the resulting equations levels of noise
are then plotted in a graph. Furthermore using the Time library to measure the amount of time it takes to run
the model as the amount of random error increases.
\subsubsection{How to mitigate noise in data: }
 Ways to mitigate the noise and it’s affects on the model were explored. Functions such as ”denoise,” in the
symbolic regression library helped to some extent. However after a certain point, such methods do not seem to
offer much assistance.\\
I also made my own denoise algorithm. I implemented various different denoise algorithms to see what effects
they had. Firstly I implemented a simple moving avergae as a way to mitigate the noise in the dataset. reword
this -¿ ” Simple and fast, smooths data well by averaging neighbors. However, it blurs sharp changes and is
sensitive to extreme outlier values, pulling the average significantly and distorting the signal.”
These were my results, this is the pseudo code, explain the alogrithm\\

The second denoise algoirthm I implemented is a median filter, and this is what effects it has, and this is how
i implemented it. Insert Pseudo code. reword: ”Excellent at removing spikes and preserving edges better than
averaging. Less affected by outliers. Can sometimes slightly distort the overall shape of the signal, especially
with large window sizes.”\\

Finally this is the third algorithm that I had implemented for denoising. Wavelet Denoising, this is the effects,
and this is the pesudo code. Reword this -¿ ”Transforms data to isolate noise, preserving both smooth and sharp
signal features effectively. More complex to understand and requires careful selection of wavelet type and pa-
rameters for optimal results, which can be tricky.”\\

\subsubsection{Modelling the noise: }

\section{Symbolic Regression from Scratch: }
\subsection{The core: }
The core and essential part of any symbolic regression model, lies in the way it at the simplest level, generates
and traverses the search tree of possible equations and expressions that may fit the data presented to it.
In order to save time, and to test if my expression generation was working as intended, i have started with sim-
ple 2 variable equaitons, and also pass in the speific operations used in the equation. Furthermore this is also
extended to handle constants and more later on.
enter in the pseudo code here.\\

Then I further improved this, by designing a resursive way to generate these expressions, to allow to generate
more robust equations from the given variables. Also this is dynamic, so it can
enter in pseudo code.\\

\subsubsection{ Exploiting Physical Properties: }

The next step is to then start to truncate these generated expressions as much as possible to prune the search
tree. One of the ways you can do this is through exploiting the symmertrical property of physical equations and
how they are mathematically equivalent. Such as removing duplicate expressions.\\
This is how I achieved that.\\
Give pseudo code here.\\

\subsubsection{Dealing with constants:}

Another way i further pruned the amount of expression, is through filtering all the expressions generated through
the newer recursive generator, by removing all the expressions that did not contain all the specified variables.
This is in order to save further time later on during the evaluation seciton.
Insert in pseudo code:

\subsubsection{Dealing with powers:}
So appling powers to expressions, I applied the power to the expression. This also allows you to prune the
search tree further by not needed to generate redundant expressions with powers.
This is the pseudo code.
Then I filtered based on if the expression contained the power, this allows me to further prune the tree. In a more
robust model, this is dervied from scratch, however for the sake of computation time, and flexibility, I decided
to proceed with this approach as it saves some time.
this is the pseudo code.
\subsubsection{Chaining powers and constants:}
The next step is to chain together powers and constants, such that both are applied to the expressions. It can
already be chained as with the design it already has. However it needs to be filtered poperly in order to maintain
the least amount of expressions possible.
Te constants filter can be used, but the powers are inside the constants, and therefore the older power filter does
not work as intended. Therefore i needed to redesign it such that it will function recursively.
This is the code.
However, sometimes this gives off constants that are chained, such as sin(sin()), and so to filter this futher, I
want to filter out expressions with more than one instant of the constant that is chained.
this is the code - filter single constant
\subsubsection{Loading data:}

Next I needed a way to load the data I has typed up. At this point I was focused on testing as quick as possible
and in order to proceed in a prompt manner I made up some dummy data values. Afterwards I made the decision
to keep the data as a numpy array, because this will be faster then a text file, there are some various reasons for
this, such as numpy arrays being stored in memory, the efficiency of the nderlying data format it is stored in
(binary), and finally numpy uses c, and so it vectorises operations, making it far faster.
Insert in pseudo code.


As you can see I check if the number of variables entered matches the shape of the arrary in X, which here is the
input data, and y being the target data, as in the final result. Ie x contains the mass and acceleration values, and
y is the array of the result of the equation f = ma, so it only contains the value of f in it. This is a basic check to
make sure the number of columns all have a corresponding variable.

\subsubsection{Evaluating expressions:}
Next I evaluate the expressions that I had generated, and i assign the variables to a column of the data, in in-
creasing order. Then this is substituted into the equation, and the expressions are run, and there is an array
of outputs of the expression. This essentially evaluates every generated expressions that has been pruned, and
returns a np arrya of the results of those expressions based on the input data.
insert in pseudo code here.
Then like the paper suggested, insert in paper here, I used a medium error description length loss function, and
have implemented it in the same way as in the paper. Using error squared, making all the errors positive, and
added 1 as a constant to ensure that all the errors are greater than the value, when taking the log.
Insert in pseudo code.
Then furthermore I also implemented 2 other loss algorithms, specifically root mean squared loss as well as
mean absolute error.
insert in pseudo code.
This was to help bridge and improve upon the loss algorithm used in the paper, as these two have their own
advantages, and a combined hybrid approach seemed smarter.
Explain why later.
\subsection{Polynomial Fit Module: }
Now that the simple, core of the algorithm works, and is adapted to take care of contants, powers, variables,
generate expressions, and filter out the redundancy using physical properties of the world such as symmtery,
I now aimed to futher extend the program by writing a polynomial fit module. The aim of the polynomial fit
technique is to.
Why: Many functions in physics (or parts of them) are low-order polynomials.
Why: It’s a computationally very cheap method for this specific function class.
How: It attempts to fit the given data to a sum of polynomial terms.
How: It generates all possible polynomial terms up to a specified low degree (e.g., degree 4).
How: For each data point, this creates a linear equation where the unknowns are the polynomial coefficients.
How: It solves the resulting system of linear equations using standard methods like least squares.
How: The Root Mean Squared Error (RMSE) of the fit is calculated.

How: If the RMSE is below a predefined tolerance (p ol), thepolynomialisacceptedasasolution.
Effect: It acts as a fast base case in the recursive algorithm, quickly solving problems that are simple polynomi-
als.
Effect: It can also solve sub-problems that are transformed into polynomials by other modules (e.g., dimen-
sional analysis, inverting the function).

\subsubsection{Data Loading:}
So to start, I began by creating the data loading function. The aim was to take in a numpy array, with the data,
along with a list of variables. Then comparing the shape of the data column and the number of variables in order
to make sure the input is sufficient.
This is the pseudo code. 


\subsubsection{Generating polynomial expressions:}

Then the next step is to generate polynomaial expressions, and then it will return a list of polynomial expres-
sions on the list.
Pseudo conclude


\subsubsection{Filtering the Polynomial expressions:}


The filtere xpressionsf unctionprogrammaticallyf ilterssymbolicexpressionsusingstructuralandsemanticconstraints.Lev
suitedf orlarge − scalesymbolicf ilteringtaskswherestrictmathematicalstructuremustbeenf orced.
insert in pseudo code.
This initial version only worked for symblic constants, ie sin, cos etc, and didn’t work for numbers, or nteger
coefficients, i caught this error during testing and I rewrote the function so that it works for integer coefficients


\subsubsection{Evaluating expressions:}

Now I need to take the filtered expressions, and try fit the model to the dataset np array. The model fitting
logic function fits polynomial expressions to input data by finding the best set of coefficients that minimize the
error between the predicted and actual output values. It tests multiple polynomial degrees (up to a specified
maximum) and selects the one that provides the lowest error, ensuring an optimal balance between accuracy
and complexity.
I use root mean squared error to calculate the loss, and the funciton returns a list of loss, per expression. 

So i take an expression, then I substitute in the variables using the input data, calculate the predicted y value of
the said equation, then take it away from the true value of y the target and then use that to calculate the rmse.
pseudo code.
Best polynomial fit:
Now that I have the list of expressions and the corresponding rmse values, i pick the lowest rmse as the most
accurate polynomial fit for the data.
insert in pseudo code . 


\subsection{Dimensional Analysis: }

Physics Constraint: Physical equations must be dimensionally consistent (units on both sides must match).
Strong Simplification: This dimensional constraint severely limits the possible forms of the unknown function.
First Step: AI Feynman applies dimensional analysis as the very first attempt to simplify the problem.
Unit Representation: Units of variables (like mass, length, time) are represented as vectors of integer powers.
Linear System: A linear system is set up based on the unit vectors of the input variables and the target variable.
Dimensionless Combinations: Solving this system and finding the null space reveals combinations of variables
that are dimensionless.
Problem Transformation: The original problem is transformed into finding a function of these new, dimension-
less variables.
Reduced Variables: This process typically reduces the number of independent variables the algorithm needs to
search over.
Search Space Reduction: A smaller number of variables drastically shrinks the combinatorial search space for
subsequent steps.
Efficiency Boost: It makes Polynomial Fit, Brute Force, and Neural Network-guided searches significantly
faster and more likely to succeed. 


\subsubsection{Handling Units:}

So i went to the ai feynman database website, and downloaded their units.csv to get a better idea of all the units
in the dataset, I was dealing with. Then i had a look at all the required units, and then made a unit table, array,
so that each unit corresponds to a unique power of the bsaic si units which I also implemented, as an array/list.
code: \\


\subsubsection{Construct Matrix and Target Vector:}

This function constructs the dimensional matrix M and target vector b essential for dimensional analysis. It
accepts lists of independent and dependent variable names and a dictionary mapping variable names (keys) to
their unit vectors (values).
Technical Implementation: Unit vectors for independent variables are retrieved via dictionary lookup, using
lowercase variable names (var.lower()) to ensure case-insensitivity. These vectors are efficiently assembled into
the columns of matrix M using numpy.columns tack.T hedependentvariable′ sunitvectorf ormsvectorb.try...exceptKeyErrorblo
Design Rationale: Case-insensitivity enhances usability. numpy.columns tackof f ersperf ormance.Expliciterrorhandlingpreven
code:\\ 


\subsubsection{Solving Dimension and Basis Units:}

This function determines the exponents for dimensional scaling and dimensionless groups by solving the lin-
ear systems Mp = b and MU = 0. It takes the dimensional matrix M and target vector b as input. The
function first converts these NumPy arrays into SymPy matrices (sp.Matrix) to leverage symbolic compu-
tation capabilities. It then attempts to find an exact particular solution p for Mp = b using SymPy’s LU-
solve method, chosen for its ability to yield rational solutions. Robust error handling via try...except ad-
dresses potential issues like inconsistent systems. Finally, it calculates the null space basis U of M using
Ms ym.nullspace(), whichidentif iesthecombinationsf ormingdimensionlessgroups.T hef unctionreturnsthesymbolicsolu
Think of it like this: you have a target physical quantity (b, like Force) that depends on several input quantities
(M, like mass, length, time).
Finding the ”Unit-Fixing” Part (p = Ms ym.LU solve(bs ym)) :
The function first figures out the specific combination of powers (p) of your input variables (x) that you need to
multiply together ( xp )sothattheresulthastheexactsamephysicalunitsasyourtargetvariable(b).
For example, if the target is Force ([M L T2]) and inputs are mass ([M]), length ([L]), and time ([T]), it would
find p corresponding to mass1 * length1 * time2.
It uses LUsolve from SymPy to try and find an exact (often simple fraction or integer) solution for these powers
p.
Finding the ”Dimensionless Combinations” (U = Ms ym.nullspace()) :
After accounting for the basic units, any remaining relationship must involve combinations of input variables
that have no units at all (they are dimensionless numbers, like Reynolds number).
The function finds all the fundamental ways (U) you can combine powers of the input variables ( xu )suchthattheunitscompletelycan
In essence, the function:
Separates the part of the formula responsible for getting the units right (p).
Identifies all the core dimensionless building blocks (U) that the rest of the formula must be made from.
This allows the main algorithm to later focus on finding the relationship between these dimensionless quantities,
which is a simpler problem than the original one involving various physical units. 



\subsubsection{Data Transformation Function:}
his function converts original physical data (datax , datay )intoadimensionlessf ormusingpreviouslycalculatedexponents(p, U ).
How it Works: First, it calculates a scalingf actorf oreachdatapointbyraisinginputvariables(datax )tothepowersspecif iedinp(u\\



\subsubsection{Symbolic Transformation Generator: }

This function generates the symbolic mathematical expressions corresponding to the dimensional analysis trans-
formation. It accepts the original independent variable names (independentv ars)andtheexponentvectorsf orscaling(p)anddimens
How it Works: It first creates SymPy symbolic objects for each input variable name using sp.symbols. Utilizing
these symbols and the scaling exponents p, it constructs the symbolic expression symbolicp representingtheunit−
f ixingscalingf actor(xp )viasp.M ul.Subsequently, ititeratesthrougheachexponentvectoruinU, buildingthecorrespondings
insert code:\\


\subsection{Neural Network Fitting: }

The next step crutical component, is using neural networks in order to simplify the expressions further.
In order to get a smotth and differentiable approximation of the function we are looking for, this is what the
neural network allows us to do.
The neural network provides a powerful, universal function approximator (fN N )capableof learningintricatepatternsdirectlyf random\\


\subsubsection{SymbolicNetwork:}

This class defines the neural network architecture used as a universal function approximator within the sym-
bolic regression framework. It inherits from torch.nn.Module, the base class for all neural network modules in
PyTorch.
Technical Implementation: The constructor (i nit) initializesthenetworkstructure,acceptingthenumberof inputf eatures(ni nput)anddef aultingtooneoutpu
Design Rationale: This multi-layer perceptron (MLP) architecture provides significant expressive power. The
Tanh activation was chosen as specified in the reference papers, offering a smooth, differentiable non-linearity
suitable for approximating complex physical functions and enabling reliable gradient computation for subse-
quent analysis steps.\\ 


\subsubsection{Preparing the data:}

This function preprocesses raw input and output data (datax , datay )intoaf ormatsuitablef orP yT orchmodeltrainingandvalidati
Technical Implementation: It begins by converting the input NumPy arrays datax anddatay intoP yT orchtensorsof typetorch.f loat3
Design Rationale: This design adheres to standard PyTorch practices. Tensor conversion is mandatory for
framework compatibility. The train/validation split is critical for monitoring model generalization and prevent-
ing overfitting. TensorDataset provides a convenient pairing of inputs and targets. DataLoader is essential for
efficient training, enabling batch processing (managing memory and potentially parallelizing computation) and
automating data shuffling, which improves model robustness and convergence.\\

\subsubsection{Training the Network:}

This function orchestrates the supervised training process for the provided PyTorch neural network model. Its
primary goal is to iteratively adjust the model’s parameters (weights and biases) to minimize the difference
between its predictions and the true target values using the training data, while monitoring performance on
unseen validation data.
Technical Implementation: The function begins by transferring the model to the specified compute device (’cpu’
or ’cuda’). It initializes the Adam optimizer, a common adaptive learning rate optimization algorithm, linking it
to the model.parameters() and setting the learningr ate.T heM eanSquaredErrorloss(nn.M SELoss), suitablef orregression, isc
Design Rationale: This structure represents a standard PyTorch training loop. Using DataLoader enables effi-
cient batch processing. The Adam optimizer provides robust convergence properties. MSE loss directly min-
imizes the squared prediction error. Explicitly setting model.train() and model.eval() ensures correct behavior of
layers like dropout or batch normalization (if present). The torch.nog rad()contextduringvalidationpreventsunnecessarygradien\\


\subsubsection{Predict Function:}

This function performs inference using a trained PyTorch model, generating output predictions for a given set of
input data. It is designed to take a trained model, input data as a NumPy array (xn umpy), andthetargetcomputationdevice(′ cpu′ or′
Technical Implementation: The function first sets the model to evaluation mode using model.eval(). This is cru-
cial as it disables layers like dropout or batch normalization that have different behaviors during training and in-
ference, ensuring deterministic output. The input NumPy array xn umpyisthenconvertedintoatorch.T ensorwithdtype =
torch.f loat32andtransf erredtothespecif ieddevice.T hecorepredictionstepoccurswithinawithtorch.nog rad() :
contextmanager.T hisdisablesgradientcalculation, signif icantlyreducingmemoryconsumptionandspeedingupcomputatio
Design Rationale: This design follows standard PyTorch inference practices. model.eval() ensures correct
prediction behavior. Tensor conversion and device handling manage data compatibility with the model. The torch.nog rad()contextoptimizesperf ormancef orinf erence.ReturningaN umP yarrayprovidesauser−
f riendlyoutputf ormat.\\ 


\subsection{Pareto Frontier Optimisation: }

The Pareto frontier provides a principled way to manage the inherent trade-off between a formula’s accuracy
and its simplicity (complexity) in symbolic regression. Instead of seeking a single ”best” formula, the goal is
to identify all Pareto-optimal formulas: those for which no other discovered formula is simultaneously simpler
and more accurate.
I am going to implement this by plotting on a 2d plane, where x represent complexity and y represents the mean
error description length. As the algorithm generates candidate formulas (through brute-force search, trans-
formations, or recursive combinations), each candidate is evaluated and potentially added to the set of points
forming the frontier.
Crucially, any candidate formula that is dominated – meaning another formula on the frontier has both lower (or
equal) complexity and lower (or equal) loss (with at least one inequality strict) – is discarded. This Pareto prun-
ing significantly reduces the search space, especially when combining solutions from subproblems, improving
computational efficiency and robustness against noise by inherently favouring simpler explanations for a given
accuracy level. The final frontier presents the user with a spectrum of optimal solutions.\\


\subsubsection{Points:}

This Point class serves as a fundamental data structure within the Pareto frontier analysis framework. Its pri-
mary purpose is to encapsulate the essential characteristics of a single candidate formula evaluated during the
symbolic regression process.
Technical Implementation: The constructor (i nit initializeseachP ointinstancewiththreeattributes:complexity(anumericalscorerepresentingthef ormula
)
Design Rationale: This class is designed to bundle the key metrics (complexity, accuracy) required for Pareto
dominance checks with the associated formula (expression). By grouping these related pieces of data into a
single object, it simplifies the management and comparison of candidate solutions. Functions operating on
the Pareto frontier can easily access the necessary complexity and accuracy attributes from each Point object to
determine if one solution dominates another, thereby facilitating the efficient maintenance of the non-dominated
set of optimal formulas.\\


\subsubsection{Pareto Point Set:}

The updatep aretop ointsf unctionmaintainsaP aretof rontierof symbolicexpressions, balancingcomplexity(modelsimplicit
The function filters the combined list to retain only non-dominated points: a point A dominates point B if A is
both no more complex and no less accurate, with at least one being strictly better. This ensures only the best
trade-offs remain.
In symbolic regression, this is critical because we aim to discover expressions that are not just accurate but
also interpretable — meaning low complexity. The Pareto frontier lets us visualize and choose among optimal
models, avoiding overfitting (too complex) or underfitting (too simple).\\ 


\subsection{Plotting: }

I wanted ways to visualise and understand in a visual sense what was happening behind the scenes in my sym-
bolic regression program. Humans are visual creatures and looking at plots offers a more powerful way to understand data rather than staring a strng of numbers or expressions, trying to manually find patterns or mis-
takes.
I decided to use plotly, instead of the usual matplotlib, as I personally think it is more asthetically pleasing and
it has a better interactive environment.\\ 


\subsubsection{Plot for Pareto front:}

Visualizing the Pareto frontier is highly beneficial for interpreting the results of symbolic regression. It provides
an intuitive graphical representation of the fundamental trade-off between model complexity and predictive
accuracy (or loss).
Code Description: The provided code utilizes the plotly.express library to generate this visualization. A Pandas
DataFrame (df) structures the data, holding columns for ’Complexity’ (x-axis), ’Loss’ (y-axis, representing inac-
curacy like MEDL), and the corresponding symbolic ’Expression’. The px.scatter function creates a scatter plot
mapping complexity against loss, crucially using the ’Expression’ data to label each point directly on the plot via
the text argument. Further customization using fig.updatet racesenhancespointvisibility(markersize, color)andlabelpositioning
Utility and Rationale: This visualization allows researchers to readily identify the set of non-dominated solu-
tions. By plotting quantitative metrics, it elucidates points where significant gains in accuracy require substantial
increases in complexity (an ”elbow” region), facilitating informed model selection based on the specific bal-
ance desired between interpretability/simplicity and predictive power. The direct labeling of points with their
expressions provides immediate context for each optimal solution discovered.\\ 

\subsection{Main method bringing it together Regressor: }

\subsubsection{main solver}

Part 1: Dimensional Analysis and Symbolic Scaling

The function first applies Dimensional Analysis (DA) to reduce the problem’s dimensionality. It computes a transformation matrix and scaling vector by solving the unit balance equation between independent and target variables. If no dimensionless groups are found, the result is a simple scaling law expressed symbolically. Otherwise, it computes and displays the dimensionless groups and the scaling part of the solution. This ensures the model operates on physically meaningful, unit-consistent quantities, simplifying the functional search space and preventing non-physical relationships. If DA completely solves the problem, the function terminates here by outputting the symbolic solution.
Part 2: Neural Network Approximation of Dimensionless Relation

When dimensionless groups exist, the function generates dimensionless data and fits a lightweight symbolic neural network to model the relationship between transformed inputs and outputs. A standard training pipeline is executed: data preparation, model instantiation, training with backpropagation, and prediction. The model’s mean squared error (MSE) is computed on the dimensionless targets to evaluate fit quality. Neural network outputs (predictions and gradients) serve as a flexible preliminary approximation, potentially capturing nonlinear relations that guide the subsequent, more rigid symbolic regression steps.
Part 3: Polynomial Candidate Generation and Filtering

Following the neural fit, the function applies Polynomial Fitting (PF) techniques to generate candidate symbolic expressions. Polynomial terms are systematically composed using provided variables, coefficients, and operators up to a specified degree. Generated expressions are filtered based on structural criteria, such as variable presence and coefficient validity, to ensure physical plausibility. Each polynomial is evaluated against the original data, and a best-fit candidate is selected according to its error score. If a candidate perfectly matches (zero error), the function outputs the discovered symbolic expression and terminates.
Part 4: Brute Force Symbolic Search and Evaluation

If no polynomial yields a perfect fit, the function initiates a Brute Force (BF) symbolic search. It exhaustively generates expressions by combining variables, constants, operators, and powers. Multiple filters (symmetry, variable relevance, power range, constant handling) systematically prune the expression set to reduce computational complexity and preserve physically meaningful candidates. The remaining expressions are evaluated against the dataset, seeking the best match based on performance metrics. This final exhaustive step ensures that even highly non-obvious symbolic relations can be discovered if they exist within the defined operator and degree space.

\subsection{Testing:}
Every module, function and file, were thoroughly tested using dedicated tests. The boundary conditions and
inserted tests to make sure the function behaved as envisioned. There was also significant robust testing func-
tions written when chaining together various techniques and models, in order to ensure everything was working
smoothly.\\


\section{ Applying the model to Biological Data: }


\section{Broder Use Cases: }


\section{Conclusion:}

\addcontentsline{toc}{section}{References}
\bibliographystyle{plain}
\bibliography{\jobname} 


\addcontentsline{toc}{section}{Appendix:}

\section{Project Planning: }


\end{document}
