\documentclass{article}
\usepackage{graphicx}
\usepackage{xcolor}
\usepackage[margin=3cm]{geometry} % margins might need to change in the future check with professor adam 
\setlength{\parindent}{0pt}
\usepackage{pgfgantt}

\usepackage{times}
\usepackage{fancyhdr,graphicx,amsmath,amssymb}
\usepackage[ruled,vlined]{algorithm2e}
\include{pythonlisting}


\begin{document}

\begin{center}
    \Large \textcolor{red}{\textbf{Electronics and Computer Science} \\[0.1cm]} 

    \large \textcolor{red}{Faculty of Engineering and Physical Sciences \\[0.1cm]} 

    \textcolor{red}{University of Southampton \\[1cm]} 

    \vspace{1cm} 

    \textcolor{red}{\textbf{\large Ashwinkrishna Azhagesh} \\[0.5cm]}

    \textbf{25/03/2025} \\[1cm] 

    \textbf{\large An AI Approach to Chaotic Physical Systems: } \\[1cm]

    \vspace{0.5cm}

    Project supervisor: \textbf{Adam Peugeot} \\[0.3cm] 
    Second examiner: \textbf{David Millard} \\[1cm]

    Progress report submitted for the award of \\[0.1cm]

    \textbf{\large Bachelors of Science} 
\end{center}

\newpage

{\Huge \textbf{Abstract}}\\[1cm]

Empirical laws are mathematical generalisations found through observing the physical world. It has taken us centuries of gathering data, keen research along with repeated experiments, and no doubt plenty of talented scientists to discover these laws. Leading us to understand everything from the mysteries that govern the collision of two objects to the shape of the path planets thread upon.\\

Recent advances in neural networks including increases in computational power permit us to train models, that replicate, fasten and automate our discovery of empirical laws. This extends to even noisy chaotic systems such as the double pendulum. Combined with white box models, symbolic regression and explanable A.I., we can peer into the "mind," of how such models, process data and conclude their observations. Human congition is inherently finite in its capacity for thought and observational ability, has been historically overcome through the development of new tools such as the microscope. Similarly, congitive biases can be mitigated, by utillising artificial intelligence, which is a rapidly emerging technology capable of expanding our perception and analysis.\\       


\newpage

\fbox{\underline{\textbf{Statement of Originality}}}
\\[0.5cm]

- I have read and understood the ECS Academic Integrity information and the University’s
Academic Integrity Guidance for Students.\\

- I am aware that failure to act in accordance with the Regulations Governing Academic Integrity
may lead to the imposition of penalties which, for the most serious cases, may include
termination of programme.\\

- I consent to the University copying and distributing any or all of my work in any form and
using third parties (who may be based outside the EU/EEA) to verify whether my work
contains plagiarised material, and for quality assurance purposes.\\

\fbox{
\underline{\textbf{You must change the statements in the boxes if you do not agree with them.}\\
}}
\\[0.5cm]

We expect you to acknowledge all sources of information (e.g. ideas, algorithms, data) using
citations. You must also put quotation marks around any sections of text that you have copied
without paraphrasing. If any figures or tables have been taken or modified from another source,
you must explain this in the caption and cite the original source.\\

\fbox{
\underline{\textbf{I have acknowledged all sources, and identified any content taken from elsewhere.}\\
}}
\\[0.5cm]

If you have used any code (e.g. open-source code), reference designs, or similar resources that
have been produced by anyone else, you must list them in the box below. In the report, you must
explain what was used and how it relates to the work you have done.\\


\fbox{
\underline{\textbf{I have not used any resources produced by anyone else.}\\
}}
\\[0.5cm]

You can consult with module teaching staff/demonstrators, but you should not show anyone else
your work (this includes uploading your work to publicly-accessible repositories e.g. Github, unless
expressly permitted by the module leader), or help them to do theirs. For individual assignments,
we expect you to work on your own. For group assignments, we expect that you work only with
your allocated group. You must get permission in writing from the module teaching staff before
you seek outside assistance, e.g. a proofreading service, and declare it here.\\

\fbox{
\underline{\textbf{I did all the work myself, or with my allocated group, and have not helped anyone else.}\\
} }
\\[0.5cm]

We expect that you have not fabricated, modified or distorted any data, evidence, references,
experimental results, or other material used or presented in the report. You must clearly describe
your experiments and how the results were obtained, and include all data, source code and/or
designs (either in the report, or submitted as a separate file) so that your results could be
reproduced.\\

\fbox{
\underline{\textbf{The material in the report is genuine, and I have included all my data/code/designs.}\\
}}
\\[0.5cm]

We expect that you have not previously submitted any part of this work for another assessment.
You must get permission in writing from the module teaching staff before re-using any of your
previously submitted work for this assessment.\\

\fbox{
\underline{\textbf{I have not submitted any part of this work for another assessment.}\\
}}
\\[0.5cm]

If your work involved research/studies (including surveys) on human participants, their cells or
data, or on animals, you must have been granted ethical approval before the work was carried
out, and any experiments must have followed these requirements. You must give details of this in
the report, and list the ethical approval reference number(s) in the box below.\\

\fbox{
\underline{\textbf{My work did not involve human participants, their cells or data, or animals.}\\
}}
\\[0.5cm]


ECS Statement of Originality Template, updated August 2018, Alex Weddell aiofficer@ecs.soton.ac.uk\\


\newpage 

{\Huge \textbf{Abstract}}\\[1cm]


I would like to thank my supervisors, Professor Adam Peugeot and Professor David Millard, for all the help and advise I received throughout this project. \\


\newpage

\tableofcontents 

\newpage
\addcontentsline{toc}{section}{Abstract}
\addcontentsline{toc}{section}{Statement of Originality}



\section{Introduction: }

\subsection{Motivation: }

As there is more data being generated than ever before and new experiments, we need a systematic and automatic way to deduce various mathematical patterns and laws in 
these data. Through the use of symbolic regression we can utilise  these data, and in an explainable manner deduce various new physical laws. In this research I have also extended this beyond physics and have applied this to biological data sets which is a novel application of this method. Perhaps extend this beyond or add a sectionsaying this can also be applied to nlp and that it can learn the rules in language and writing etc. \\ 

Talk a little about the way this is used outside of this niche use case, and in research, so of course I need to look and research into this.\\


\section{Previous Work: }

\subsection{Literature Review: }


\section{Noise: }

In this section, I aimed to explore how noise affects the model, and potential ways to mitigate it. Continuing onwards from the previous model, in the data generation step, noise was artificially added, and the results were observed.\\

\subsection{How noise affects the model: }

So in order to add noise to the generated data set, I imported in random, and used the randn.int function. In order to vary the inputs, another function was created that incrememntally passes in higher numbers as parameters to the random function, allowing each set of generated data to incrememntally become more and more noisy. Then the symbolic regression model is run on these new data sets, and the resulting equations levels of noise are then plotted in a graph. Furthermore using the Time library to measure the amount of time it takes to run the model as the amount of random error increases. \\ 


\subsection{How to mitigate noise in data: }

Ways to mitigate the noise and it's affects on the model were explored. Functions such as "denoise," in the symbolic regression library helped to some extent. However after a certain point, such methods do not seem to offer much assistance.\\



\section{Predicting future states using initial conditions: }

One of the use cases of such AI models, is to infact predict the future states and values of chaotic systems. This has been applied to find approximate solutions for various initial states for complex problems such as the 3 body problem(check this!). It will be applied here on the simplest chaotic system, which is a double pendulum. Using a neural network and training data, we will explore how the prediction works with varying levels of initial force and conditions.\\


\subsection{ Modelling the noise: }

So in order to model the noise, I used the python random library, and generated random numbers between 0 and an ever increasing amount of randomness, in oder to guage the accuracy as noise increased for the model. I was also part


%\subsection{ Using Neural Netwroks to approximate the functions: }

%\subsection{ SciNet: }

%\subsection{ Prediction of future states: }

%\subsection{ Varying Initial Conditions: }


\section{Neural networks to solve the Feyman Equations: }


\subsection{ Modelling the Network: }

\subsection{ Generating the data: }

\subsection{ Using C to Brute Force Solutions: }


\section{ Applying the model to Biological Data: }


\section{Broder Use Cases: }


\section{Conclusion:}

\addcontentsline{toc}{section}{References}
\bibliographystyle{plain}
\bibliography{\jobname} 


\addcontentsline{toc}{section}{Appendix:}

\section{Project Planning: }


\end{document}
