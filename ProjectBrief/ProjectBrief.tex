\documentclass [14pt]{article}
\usepackage{graphicx} % Required for inserting images
\usepackage{geometry}
\usepackage{pdfpages}
\usepackage{url}
\usepackage{parskip}
\usepackage{listings}
\geometry{
	tmargin=15mm,
	lmargin=15mm,
	rmargin=15mm,
	bmargin=15mm
}
\title{Project Brief - An AI approach to Chaotic Physical Systems:}
\author{
	Azhagesh Ashwinkrishna\\
	\texttt{aa9g22@soton.ac.uk}
}
\date{October 08, 2024}

\begin{document}

\maketitle 

\newpage 

\section{Brief: }

It took humans centuries to derivie physical laws, can this process be sped up through AI, by feeding it data and letting the model derive complex laws for us. I aim to derive physical laws, from experimental data. I will explore deriving simpler physical laws such as acceleration without air resistance, and move onto to complex chaotic systems such as pendulums, and explore how an unbiased AI views the physical world, compared to humans who's views of physical systems are naturally biased through systematic learning. 

Can this lead to perhaps different prespectives of viewing the physical world around us, allowing for further progress? 


\subsection{Scope and Goals: }

 - Aim to derive simple laws of motions (ie acceleration) through AI frameworks.\\ 

- Move onto more complex systems such as pendulums, and initially explore smaller initial values, moving onto larger initial values, 
thereby increasing the chaos, and difficulty of spotting patterns. \\

- To explore using various AI techniques, (Graph Neural Networks, Deep learning, Neural Networks) in combination with no prior knowledge and observe how and in what form the physical laws are derived.  
- Simulate physical data required using pymunk, and perhaps use real world data from physics labs.\\ 



\end{document}
